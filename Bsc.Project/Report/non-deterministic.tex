\chapter{پیچیدگی ارتباطی غیرقطعی}\label{chapter2}

در این بخش، سعی بر این است که مشابه تئوری پیچیدگی محاسباتی، مدل غیرقطعی\footnote{Non-deterministic} را در پیچیدگی ارتباطی نیز تعریف کنیم.  مطالب این فصل با اقتباس از 
 \cite{ 
 		nissan09,
 		lee09,
 		tim15,
 		toni14,
 		sherstov18,
 		laszlo86
 		}
 تهیه شده است.

برای هر تابع بولی $f: X \times Y \rightarrow \{0, 1\}$، یک پروتکل غیرقطعی دو بخش خواهد داشت. در ابتدا، یک پیشگو\footnote{Oracle} که به ورودی هر دو طرف دسترسی دارد یک رشته $a$ را به هر دو طرف می‌دهد. در مرحله دوم، بازیکنان با در اختیار داشتن این رشته و ورودی‌های خودشان، پروتکل را مانند قسمت قطعی ادامه می‌دهند و در مورد مقدار تابع تصمیم می‌گیرند. اگر خروجی را با $\Pi(x, y, a)$ نشان دهیم،  این پروتکل مقدار $f$ را به درستی محاسبه می‌کند اگر:
\[f(x, y) = 1\ \Rightarrow \exists a, \Pi(x, y, a) = 1\]
\[f(x, y) = 0\ \Rightarrow \forall a, \Pi(x, y, a) = 0\]
هزینه این پروتکل، جمع حداکثر طول رشته $a$ و حداکثر تعداد بیت‌های مخابره شده بین دو طرف است. هزینه غیرقطعی محاسبه $f$ که با $N^1(f)$ نشان داده می‌شود، حداقل هزینه پروتکلی‌ست که $f$ را محاسبه می‌کند. متناظر با این تعریف، هزینه co-nondeterministic برای تابع $f$ به همین شکل بیان و با $N^0(f)$ نشان داده می‌شود.
   
\begin{example}
تابع نامساوی را درنظر بگیرید. تعریف رسمی این تابع به شکل زیر می‌باشد:
\begin{equation}
    NEQ_{n}(x,y) =
    \begin{cases}
        0 & \text{ x=y}\\
        1 & \text{در غیر این صورت}
    \end{cases}
\end{equation}
و ماتریس مشخصه آن نیز به این شکل است:
\begin{center}
    $\begin{bmatrix}
         0 & 1 & \cdots & 1 \\
         1 & 0 & \cdots & 1 \\
         \vdots & \vdots & \ddots & \vdots \\
         1 & 1 & \cdots & 0
    \end{bmatrix}$
\end{center}
هزینه محاسبه غیرقطعی این تابع ($(N^1(NEQ$) برابر $O(\log n)$ است. برای مشاهده این موضوع، دقت کنید که اگر دو ورودی متفاوت باشند، پیشگو می‌تواند شماره بیت اختلاف را به عنوان رشته اولیه به هر دو طرف ارسال کند. در نتیجه، هزینه کل پروتکل برابر طول رشته کمک اولیه ($O(\log n)$) و هزینه چک کردن آن بیت خواهد بود. (آلیس بیت مشخص شده را برای باب ارسال می‌کند، باب مشاهده می‌کند که رشته‌ها متفاوت هستند. این عمل به یک بیت مخابره نیاز دارد.)\\
در جهت دیگر، مشاهده کنید که اگر دو ورودی با هم مساوی باشند، هیچ رشته اولیه‌ای وجود ندارد که پیشگو بتواند در ابتدا به هر دو طرف ارسال کند و فرآیند چک کردن تساوی را آسان‌تر نماید. پس برای این تابع، 
$N^0(NEQ) = O(n)$
خواهد بود.
\end{example}
\section{پوشش‌ها}

مشابه ارتباط پیچیدگی قطعی با مستطیل‌ها، پیچیدگی غیرقطعی ارتباط مستقیمی با پوشش‌ها\footnote{Covers} دارد.

\begin{definition}
برای هر $z \in \{0, 1\}$، یک $z$-پوشش\footnote{z-Cover} برای $f$، مجموعه‌ای از مستطیل‌های $R_1, \dots, R_N$ است که ممکن است اشتراک داشته باشند، به طوری که 
$f^{-1}(z) = \cup R_i$.
حداقل سایز یک z-پوشش برای $f$ را با $C^z(f) = N$ نشان .می‌دهیم
\end{definition}
\begin{theorem}
برای $z \in \{0, 1\}$ داریم: $N^z(f) = \log C^z(f) + O(1)$ به عبارت دیگر، ارتباط هزینه مخابره غیرقطعی با پوشش‌ها معادل مخابره قطعی با  مستطیل‌هاست.
\end{theorem}

\section{کلاس‌های پیچیدگی}

یکی از بزرگ‌ترین مسائل باز موجود در پیچیدگی محاسباتی، درستی تساوی $\mathbf{P} = \mathbf{NP} \cap \mathbf{coNP}$ است. در این قسمت، سعی می‌کنیم تعاریف مشابهی برای کلاس‌های متناظر آن‌ها بیان کنیم و این تساوی را در پیچیدگی ارتباطی اثبات کنیم.
\begin{definition}
 برای یک تابع دودویی $f: \{0, 1\}^n \times \{0, 1\}^n \rightarrow \{0, 1\}$ داریم:

\begin{itemize}
    \item $D(f) = polylog(n) \ \Rightarrow\ f \in \mathbf{P}^{CC}  $
     \item $N^1(f) = polylog(n) \ \Rightarrow\ f \in \mathbf{NP}^{CC}  $
     \item $N^0(f) = polylog(n) \ \Rightarrow\ f \in \mathbf{coNP}^{CC}  $
\end{itemize}

که در آن $polylog(n) = O(\log^c n)$ است.
\end{definition}
\begin{lemma} \label{lem:1}
 برای هر تابع دودویی، $D(f) = O(N^0(f)N^1(f))$
\end{lemma}
\begin{theorem}
\begin{equation*}
P^{CC} = NP^{CC} \cap coNP^{CC}
\end{equation*}
\end{theorem}
\begin{proof}
مشاهده جهت اول، مشخص است. مجموعه 
$P^{CC}$
زیر مجموعه هر دو مجموعه
$NP^{CC}$ و $coNP^{CC}$
است و پس زیرمجموعه اشتراک آن‌ها نیز خواهد بود. \\
برای اثبات قسمت دوم، مشاهده می‌کنیم که از آن‌جایی که طبق تعریف، اگر تابعی هم در 
$NP^{CC}$ و هم در $coNP^{CC}$
باشد، پس 
$N^1(f) = O(\log^{c_1} n)$ و 
$N^0(f) = O(\log^{c_2} n)$
خواهد بود. حال از لم \autoref{lem:1} استفاده می‌کنیم. 
\[D(f) = O(N^0(f)N^1(f)) \Rightarrow D(f) = O(N^0(f)N^1(f)) = O(\log^{c_1} n \  \log^{c_1} n) = O(\log^c n)\]
\end{proof}
\section{تکنیک‌های کمینه‌یابی}

تکنیک‌های کمینه‌یابی برای پیچیدگی غیرقطعی به طور مستقیم از لم قبل نتیجه می‌شوند.
\[N^1(f) \geq \Omega(\frac{D(f)}{N^0(f)})\]
و یا از معادل آن استفاده می‌کنند:\\
\begin{lemma}
 برای هر تابع دودویی، نتایج زیر برقرار است:
 \[D(f) = O(N^1(f)\.\log (\mathrm{rank}(M_f))) \Rightarrow N^1(f) \geq \Omega(\frac{D(f)}{\log (\mathrm{rank}(M_f))})\]
 \[D(f) = O(N^0(f)\.\log (\mathrm{rank}(M_f))) \Rightarrow N^0(f) \geq \Omega(\frac{D(f)}{\log (\mathrm{rank}(M_f))})\]
\end{lemma}
\section{نتیجه‌گیری}

تا به حال، چندی از ملاک‌های سنجش پیچیدگی ارتباطی را بررسی کردیم. ارتباط کلی آن‌ها به شکل زیر است:
\begin{equation}
2^{\Theta(\sqrt{D(f)})} \leq C(f) \leq C^D(f) \leq C^P(f) \leq 2^{\Theta(D(f))}
\end{equation}
\begin{itemize}
    \item $C^P(f)$ کم‌ترین تعداد برگ‌های درخت پروتکل برای تابع $f$ است.
    \item $C^D(f)$ کم‌ترین تعداد مستطیل‌ها در یک تقسیم‌بندی برای تابع $f$ است.
    \item $C^z(f)$ حداقل سایز یک z-پوشش برای تابع f است.
    \item $C(f) = C^0(f) + C^1(f)$
\end{itemize}

نامساوی اول مستقیما از لم 1 نتیجه می‌شود. دو نامساوی اول نیز در فصل قبل اثبات شده‌اند.
