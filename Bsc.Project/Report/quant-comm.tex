\chapter{پیچیدگی ارتباطی کوانتومی}
در فضای پیچیدگی ارتباطی، آیا می‌توان از مکانیک کوانتومی به نحوی بهره برد که پیچیدگی مخابره نیز کاهش یابد؟ قضیه هولف\footnote{Holevo's Theorem} نشان می‌دهد که اگر بنا باشد که فقط $n$ کیوبیت بین دو نفر جابه‌جا کنیم، امکان ندارد که بین آن دو بتوانیم چیزی فراتر از $n$ بیت کلاسیک جا‌به‌جا کنیم؛ مگر این که کیوبیت‌های دو طرف درهم‌تنیده باشند. 
\cite{holevo73}
لبته در این صورت هم فقط می‌توان دوبرابر تعداد کیوبیتی که می‌فرستیم، بیت ارسال کنیم. در یک تناقض آشکار، مسائلی وجود دارد که الگوریتم‌های توزیع شده کوانتومی، به صورت بهینه توسط سیستم کلاسیک شبیه‌سازی نمی‌شوند. در این بخش، تفاوت اطلاعات و محاسبات توزیع شده برای یک سیستم با تنظیمات کوانتومی، چه به صورت کانال کوانتومی و چه به صورت استفاده از کیوبیت‌های درهم‌تنیده را نشان ‌می‌دهیم و پروتکل‌های کوانتومی را برای چند مساله مطرح معرفی می‌‌کنیم. 
مطالب این بخش از
 \cite{wolf19, Gilles01} اقتباس شده است. 
\section{یک سوال کوانتومی}
تصور کنید که آلیس و باب مسائل پیچیدگی ارتباطی، این اجازه را داشته باشند که در یک کانال کوانتومی و با ارسال کیوبیت‌ها با هم ارتباط برقرار کنند و یا این که قبل از شروع مخابره، با هم یک حالت کوانتومی درهم‌تنیده را تقسیم کنند. 

برای مدل اول، فرض کنید که حالت بین دو طرف از سه قسمت تشکیل شده است، فضای مخفی آلیس، کانال و فضای مخفی باب. حالت شروع اولیه را برای یک تابع مانند  $f: X \times Y \to \{0,1\}$ به صورت $\ket{x}\ket{0}\ket{y}$ فرض کنید که در آن $x \in X$ ورودی آلیس است و $y \in Y$ ورودی باب و کانال در حالت اولیه $\ket{0}$ قرار دارد. حال آلیس می‌تواند محاسبات و انتقال پیام روی کانال را با استفاده از عملگرهای یکانی بر روی حالت خودش و حالت کانال انجام دهد و باب به همین ترتیب. در انتهای پروتکل، آلیس یا باب با انجام یک اندازه‌گیری، نتیجه پروتکل را اعلام می‌کند.  \cite{yao93} 

در مدل دوم، آلیس و باب تعداد نامحدودی کیوبیت درهم‌تنیده با هم به اشتراک می‌گذارند، ولی در ادامه پروتکل از یک کانال کلاسیک با قابلیت ارسال بیت معمولی استفاده می‌کنند. برای محاسبه پیچیدگی، تنها بیت‌های ارسال شده را می‌شمریم و نه تعداد ای-بیت‌ها را. پروتکلی با این سیستم را می‌توان با استفاده از پروتکل مدل اول با سربار $2$ برابر شبیه‌سازی کرد، مانند کاری که در فرابرد کوانتومی کردیم. توجه کنید که یک بیت درهم‌تنیده می‌تواند حکم یک بیت تصادفی مشترک بین آلیس و باب را داشته باشد. 
\cite{cleve97}

مدل سوم، از قدرت هر دو مدل استفاده می‌کند: آلیس و باب تعداد بی‌نهایتی کوبیت در هم تنیده در اشتراک دارند و از یک کانال کوانتومی برای انتقال اطلاعات استفاده می‌کنند. توجه کنید که این مدل هم با یک سربار با ضریب 2 معادل حالت دوم است، با استفاده از فرابرد کوانتومی. 

حال سوالی که مطرح می‌شود این است: آیا به بهبودی دست می‌یابیم؟ طبق مقدمه، قضیه هولف نشان می‌دهد اطلاعات کوانتومی مخابره شده فراتر از اطلاعات کلاسیک مخابره شده نخواهد بود مگر آنکه یک حالت درهم‌تنیده داشته باشیم. ولی مساله‌ای که با آن روبرو هستیم، مساله مخابره اطلاعات کامل نیست. در واقع آلیس یا باب علاقه‌مند به ورودی طرف دیگر نیستند، بلکه هدف این کار محاسبه یک تابع $f(x,y)$ با خروجی $1$ بیت است. مفهومی که مرز مخابره و محاسبه توزیع شده را جدا می‌کند، در مثال‌های زیر به خوبی نمایش داده شده است. 

\section{الگوریتم دوچ-جوزا: توزیع شده}\label{djd}
اولین فاصله پیچیدگی ارتباطی کوانتومی و کلاسیک، در همتای ارتباطی و توزیع شده الگوریتم پرس‌وجوی دوچ-جوزا مطرح شد.
\cite{cleve98}
در این مساله، آلیس و باب هر کدام یک رشته $n$ بیتی دارند که برای این ورودی وعده‌ای به ما داده شده است. توجه کنید که این مساله، نوع وعده‌داده‌شده\footnote{Promise Problem} از مساله برابری است. وعده مذکور به شرح زیر است:

\begin{center}
	مساله دوچ-جوزا توزیع شده: یا $x=y$ و یا $x$ و $y$ دقیقا در $n/2$ بیت‌ها با هم اختلاف دارند ($n$ سایز ورودی‌های $x$ و $y$ است).
\end{center}
حال پروتکلی را مطرح می‌کنیم که با استفاده از
 $\log(n)$ کیوبیت مساله را حل می‌کند: 
\begin{enumerate}
	\item آلیس برای باب حالت
			$\log{n}$-
				کیوبیتی 
				$\frac{1}{\sqrt{n}}\sum_{i=1}^{n}(-1)^{x_{i}}\ket{i}$ 
				را ارسال می‌کند. این حالت با استفاده از عملگر یکانی $H$ و حالت اولیه
				 $\ket{0...0}$ و عملگر یکانی $T_{f}$ به دست می‌آید. 
		
		\item  باب هم عملگر $\ket{i} \to (-1)^{y_{i}}\ket{i}$ را اعمال می‌کند و سپس عملگر هادامارد را. سپس اندازه‌گیری انجام می‌دهد.
		\item اگر خروجی اندازه‌گیری برابر با $\ket{0^{\log{n}}}$ بود، خروجی $1$ می‌دهد و در غیر این صورت خروجی $1$.
\end{enumerate}
توجه کنید که تنها
 $\log{n}$ کیوبیت مخابره شده است. همچنین، برای فهم درستی قضیه به این توجه کنید که حالتی که در نهایت باب اندازه‌گیری می‌کند، در حالت برهم‌نهی شده زیر است: 
\begin{equation}
	H^{\otimes log{n}}(\frac{1}{\sqrt{n}} \sum_{i=1}^{n} (-1)^{x_{i}+y_{i}}\ket{i}) = \frac{1}{n}  \sum_{i=1}^{n} (-1)^{x_{i}+y_{i}} +  \sum_{j \in \{0,1\}^{log{n}}} (-1)^{i.j}\ket{j}
\end{equation}
 تنها مساله‌ای که در مورد عبارت فوق باید به آن توجه کنیم آن است که ضریب پایه $\ket{0^{log{n}}}$ در هر حالت برابر با چند می‌شود. با کمی بررسی متو‌جه می‌شویم که این ضریب برابر با 
 $\frac{1}{n}\sum_{i=1}^{n}(-1)^{x_{i}+y_{i}}$ 
 که برابر با یک خواهد بود اگر و تنها اگر $x=y$. ادامه درستی مساله مانند فصل پیشین است. 
 
 این پیشرفت در مخابره درحالی است که اگر می‌خواستیم این عملیات را با مخابرات کلاسیک و قطعی انجام دهیم، پیچیدگی معادل با $O(n)$ می‌داشتیم. 
 
 
  \subsection{مساله اشتراک}

  اگر آلیس و باب هر کدام یک ورودی به اندازه $n$ بیت داشته باشند، که $x$ ورودی آلیس و $y$ ورودی باب باشد، هدف از مخابره یافتن $i$ است که $x_{i} = y_{i}$. تلاش می‌کنیم بر اساس الگوریتم جست‌وجوی گروور، یک پروتکل بهینه برای این مساله ارائه دهیم. 
  \cite{cleve98}
  لازم است ابتدا بدانیم ورودی مساله جست‌وجوی ما چیست.
  
    قرار دهید که 
  \begin{equation}
  	z_{i} = x_{i} \wedge y_{i}.
  \end{equation}
   در نتیجه مساله جست‌وجو برابر با یافتن $i$ی خواهد بود که در آن
    $z_{i} = 1$.
   ایده اصلی آن است که آلیس و باب با کمک به هم الگوریتم گروور را اجرا کنند. نکته‌ای که نیاز به همکاری در آن حس می‌شود، عملگر یکانی $T_{z}$ است. آلیس برای این که بدون آن که اطلاعاتی به صورت بیتی به باب در مورد ورودی خودش بدهد، بتواند عملگر $T_{z}$ را روی یک حالت مانند $\ket{\phi}$ اجرا کند، که 
    \begin{equation}
    	\ket{\phi} = \sum_{i=1}^{n} \alpha_{i}\ket{i}
    \end{equation}
    لازم است از یک کیوبیت کمکی استفاده کند و توجه کنید که حالت $\ket{\phi}$ یک حالت $\log{n}$ کیوبیتی است. آلیس یک کیوبیت با مقدار اولیه $0$ در کنار $\log{n}$ کیوبیت دیگر قرار می‌دهد و سپس عملگر $O_{x}$ را روی آن اجرا می‌کند:
    \begin{equation}
    \ket{\phi} \to^{ \otimes \ket{0}} \ket{\phi}\ket{0} \to^{O_{x}} T_{x}\ket{\phi}\ket{0} = \sum_{i=1}^{n} \alpha_{i}\ket{i}\ket{x_{i}}
    \end{equation}
    سپس آلیس این
     $\log{n} + 1$ کیوبیت را برای باب می‌فرستد. 
    
    باب عملگر یکانی زیر را اعمال می‌کند و سپس نتیجه را برای آلیس می‌فرستد. 
    \begin{equation}
    	\ket{i}\ket{x_{i}} \to (-1)^{x_{i}\wedge y_{i}}\ket{i}\ket{x_{i}}
    \end{equation}
    آلیس آخرین کیوبیت را برابر با $\ket{0}$ می‌گذارد (چون $x$ را دارد این عمل یکانی است). پس در نتیجه، آلیس $T_{z}\ket{\phi}$ را دارد. در نتیجه، آلیس و باب می‌توانند با $O(\log{n})$ کیوبیت مخابره، یک بار اعمال $T_{z}$ را شبیه‌سازی کنند. برای اجرای کامل الگوریتم جست‌وجو، نیاز به $\sqrt{n}$ مرحله مخابره داریم که در نهایت پیچیدگی کوانتومی 
    $O(\log{n}\sqrt{n})$ برای این مساله نشان داده می‌شود. درحالی است که این مساله حتی در حالت احتمالاتی پیچیدگی کلاسیک $O(n)$ را دارد. 