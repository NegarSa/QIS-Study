\chapter{مقدمه}\label{intro}
مباحث نظری مربوط به پیچیدگی ارتباطی، پس از طرح اولیه آن‌ها توسط Yao در سال 1979، در حیطه‌های مختلفی استفاده شده‌اند. 
	این کاربرد‌ها، در زمینه‌‌های متنوعی مانند الگوریتم‌های جریانی، نظریه بازی‌ها‌‌ طراحی مدارهای VLSI و ... یافت می‌شوند.
	در نتیجه، این کاربردهای متنوع تحقیقات زیادی را در این زمینه شامل شده است.\\
	
محاسبات کوانتومی، مبحث جدیدی در علوم کامپیوتر است که با وجود عمر کوتاه، تغییرات زیادی را در تصور و درک ما نسبت به نظریه محاسبات و پیچیدگی محاسباتی ایجاد کرده است. 
طراحی و انتشار این الگوریتم‌ها - مانند الگوریتم شور \footnote{Shor} برای به‌دست آوردن عوامل اول یک عدد دلخواه در زمان بهینه - محققان را به پرسش این سوال دعوت می‌کند که آیا در مباحث و حوزه‌های دیگر پیچیدگی محاسباتی نیز، امکان یافت الگوریتم‌های بهینه‌تری وجود دارد یا خیر؟\\

در ادامه، پس از معرفی این حوزه‌ها و مباحث نظری پایه‌ای موردنیاز، به بررسی چند مسئله به شکل بالا می‌پردازیم و تلاش می‌کنیم با شبیه‌سازی آن‌ها، از پیچیدگی و زمان اجرای آن‌ها مطلع شویم.\\


مباحث نظری مربوط به پیچیدگی ارتباطی، پس از طرح اولیه آن‌ها توسط Yao در سال 1979، در حیطه‌های مختلفی استفاده شده‌اند. 
	این کاربرد‌ها، در زمینه‌‌های متنوعی مانند الگوریتم‌های جریانی، نظریه بازی‌ها‌‌ طراحی مدارهای VLSI و ... یافت می‌شوند.
	در نتیجه، این کاربردهای متنوع تحقیقات زیادی را در این زمینه حاصل کرده است.\\
	
محاسبات کوانتومی، مبحث جدیدی در علوم کامپیوتر است که با وجود عمر کوتاه، تغییرات زیادی را در تصور و درک ما نسبت به نظریه محاسبات و پیچیدگی محاسباتی ایجاد کرده است. 
طراحی و انتشار این الگوریتم‌ها - مانند الگوریتم شور \footnote{Shor} برای به‌دست آوردن عوامل اول یک عدد دلخواه در زمان بهینه - محققان را به پرسش این سوال دعوت می‌کند که آیا در مباحث و حوزه‌های دیگر پیچیدگی محاسباتی نیز، امکان یافت الگوریتم‌های بهینه‌تری وجود دارد یا خیر؟\\

در این پروژه، پس از معرفی این حوزه‌ها و مباحث پایه‌ای موردنیاز، به بررسی چند مسئله نظری به شکل بالا می‌پردازیم و در نهایت تلاش می‌کنیم با شبیه‌سازی آن‌ها، از پیچیدگی و زمان اجرای آن‌ها مطلع شویم.\\

بررسی کمینه‌ها در مسائل پيچيدگي ارتباطی كوانتومي در حوزه ساختمان داده‌ها
فصل‌های دوم تا چهارم، به مباحث تئوری بنیادین پیچیدگی ارتباطی - پیچیدگی قطعی، غیرقطعی و تصادفی - می‌پردازند. سپس در فصل پنجم، کاربرد خاصی از آن‌ها در حالت کلاسیک مطرح شده و نتایخ به مشاهده‌شده را برای به‌دست آوردن نتایجی در آن زمینه‌ها بیان خواهند شد.\\

از فصل ششم، مکانیک کوانتومی معرفی خواهد شد. این فصل مقدمه تاریخی و نظری روی اصول مکانیک کوانتومی و پدیده‌های غیرمنتظره آن‌هاست. در این فصل، دانش کلی‌ای در مورد جبرخطی و ریاضیات موردنیاز فرض شده است و به همین دلیل، در پیوست اول مقدمه‌ای بر مبانی ریاضی مکانیک کوانتومی آورده شده است.\\

در دو فصل بعد، الگوریتم‌های کوانتومی را بررسی می‌کنیم. فصل هفتم در مورد الگوریتم‌های معروف را مطرح می‌کند و فصل هشتم، حالت توزیع‌شده و ارتباطی آن‌ها را بررسی خواهد کرد.
در نهایت، فصل آخر شبیه‌سازی چندی از این الگوریتم‌ها را دربرخواهد داشت.