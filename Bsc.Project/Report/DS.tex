\chapter{کاربرد پیچیدگی ارتباطی در ساختمان داده‌ها }\label{chapter4}
در این بخش از گزارش، در مورد استفاده از پیچیدگی ارتباطی برای یافتن کمینه‌ها در حوزه ساختمان‌ داده‌ها می‌پردازیم. برای شروع، تعریفی بر روی مدل جست‌وجوی سلول خواهیم داشت، ساختمان‌ داده‌های ایستا و پویا را تعریف می‌کنیم و ارتباط پیچیدگی ارتباطی نامتقارن و مدل جست‌وجوی سلول را مطرح می‌کنیم. در ادامه، با معرفی تکنیک‌ حریصانه در پیچیدگی ارتباطی، کمینه‌های ساختمان داده‌ای را به پیچیدگی مخابرات نزدیک می‌کنیم. 


\section{مدل جست‌وجوی سلول}
می‌توانید مدل جست‌جوی سلول\footnote{Cell Probe} را یک پردازنده و یک حافظه دسترسی تصادفی در نظر بگیرید. یک ساختمان داده را فرض کنید که در این حافظه ذخیره شده است و پردازنده باید با استفاده از این داده به یک جست‌وجو پاسخ دهد. برای ادامه بحث، $s$ تعداد خانه‌های حافظه و $w$ تعداد بیت‌ در هر خانه است. پیچیدگی کار یه تعداد خانه‌هایی است که پردازنده در حین پاسخ به جست‌وجو به آن‌ها دسترسی می‌یابد.\cite{ajtai88} 

این مدل، اصلی‌ترین مدل برای اثبات کمینه‌ها در ساختمان داده‌هاست. این بدین معناست که کمینه‌های این مدل، کمینه‌های مدل‌های دیگر را در بر دارد. این موضوع برای بیشینه‌ها نیز صادق است. 

\begin{definition}
یک مساله ساختمان داده ایستا، مساله‌ای است که با توجه به یک $map$ مانند $f:Q\times D \rightarrow A$ است که در آن $D$ یک مجموعه از داده‌ها است، $Q$ یک مجموعه از جست‌وجوهاست و $f(q,d)$ پاسخ به جست‌وچوی $q$ در مورد داده ‏$d$ است. $A$ نیز مجموعه همه پاسخ‌هاست. 
\end{definition}

\begin{definition}
پاسخ به یک مساله ساختمان داده ایستا در مدل جست‌وجوی سلول با پارامترهای $s$ و $w$ و $t$، یک روش برای کدگذاری هر داده مانند $d \in D$ در حافظه ماشین است در صورتی که از $s$ خانه حافظه با $w$ بیت استفاده کنیم در حالی که حداکثر لازم است $t$ خانه را بررسی کنیم تا بتوانیم پاسخی بیابیم. 
\end{definition}

\begin{example}
با در دست داشتن یک مجموعه مرتب مانند $U = \{ 1,...,m\}$ و یک زیر مجموعه مانند $S \subseteq U$، به نحوی $S$ را کدگذاری کنیم و در حافظه قرار دهیم که بتوانیم اولین نیاکان یک عضو مانند $x \in U$ را در $S$ بیابیم. 
\end{example}
 
\begin{definition}
ساختمان داده‌های پویا:
ساختمان داده‌های پویا شباهت زیادی به ساختمان‌داده‌های ایستا دارند به جز آن‌که در این مسائل باید بتوان داده‌ها به صورت بهینه در ساختمان‌ داده ذخیره کرد.
\end{definition}


\section{پیچیدگی ارتباطی نامتقارن}

در این مساله، با پیچیدگی ارتباطی قطعی سروکار داریم. می‌توان هر مساله ساختمان داده را به یک مخابره تبدیل کرد: آلیس یک جست‌وجو مانند $q \in Q$ دریافت می‌کند. و باب یک داده $d \in D$ را دریافت می‌کند و هردو تلاش می‌کند که تابع $f(q,d)$ را محاسبه کنند.\cite{Nisan98} البته بین این مدل و مدل اصلی مخابره تفاوتی وجود دارد. مثلا، اندازه داده خیلی بیشتر از جست‌وجو است در نتیجه مخابرات به صورتی نامقارن است، یعنی بار ارتباطی سمتی از مخابره همواره بیشتر از سمتی دیگر است.  همچنین تنها آلیس نیاز دارد که پاسخ را بداند. در ادامه، پیچیدگی ارتباطی این مدل باید به طور دیگری سنجیده شود. آلیس پیغام خود را از مجموعه $\{1,...,s\}$ انتخاب می‌کند و باب پیغام خود را از بین $\{1,...,2^{w}-1\}$ انتخاب می‌کند. پیچیدگی پروتکل همان ماکسیمم تعداد دورهای مخابره بین آلیس و باب است. به صورت رسمی:

\begin{definition}
می‌گوییم یک پروتکل مانند $P$، یک $[t,(a,b)]^{A}$-پروتکل برای تابع $f$ است اگر همه پیام‌های آلیس حداکثر $a$  بیت باشد و پیغام‌های باب حداکثر $b$ بیت باشد و حداکثر $t$ دور مخابره انجام شود در حالی که مخابره را آلیس شروع کرده است. یک $[t,(a,b)]^{B}$-پروتکل به همین صورت تعریف می‌شود با تفاوت آن‌که باب مخابره را شروع می‌کند.  
\end{definition}

\begin{theorem}
اگر یک پاسخ به مساله جست‌وجوی سلول از $s$ خانه استفاده کند و حداکثر شامل $t$ بار دسترسی به خانه‌های حافظه باشد، یک $[t,(\log s,w)]^{A}$-پروتکل برای مساله متناظر معادل وجود دارد. 
\end{theorem}

اثبات شبیه‌سازی مطرح شده در تعریف است. \\

از قضینه بالا می‌توان نتیجه گرفت برای آن که بتوانیم یک کمینه مانند $\Omega(f)$ برای یک مساله ایستا ثابت کنیم (فرض می‌کنیم که $s \in O(n)$) باید نشان دهیم که هیچ یک $[o(f),(\log n,w)]^{A}$-پروتکلی برای مساله ارتباطی معادلش وجود ندارد. 
 

در مرحله بعد، نشان می‌دهیم که یک مساله ایستا می‌تواند به یک مساله پویا کاهش یابد.


\begin{definition}
مساله ساختمان داده ایستا $f$ متناظر با مساله پویا $g$ به این صورت تعریف می‌شود: برای یک پارامتر $d$، $f: Q \times D \rightarrow A$ تابع ایستای ماست که $Q$ همان جست‌وجوهای مساله پویا است و $D$ تمامی حالت‌های قابل دسترسی از حالت اولیه است که حداکثر $d$ بار تغییر یافتن بر روی آن رخ داده‌است.
\end{definition}

\begin{theorem}
اگر یک پاسخ مساله جست‌وجوی سلول از $n^{O(1)}$ خانه حافظه استفاده کند و $t$ بار دسترسی به حافظه انجام دهد برای تابع $g$ که یک مساله پویاست، در نتیجه مساله معادل ایستای $f$ یک پاسخ دارد که از $O(dt)$ خانه حافظه استفاده می‌کند و $O(t)$ دسترسی انجام می‌دهد. 

\end{theorem}

\begin{proof}
پاسخ ایستا به صورت ساده پاسخ پویا را شبیه‌سازی می‌کند. فرض کنید که ساختمان داده مساله پویا به نحوی در حافظه کدگذاری شده‌است که به طوری که حالت اولیه همه خانه‌ها برابر با $0$ هستند. در پاسخ مساله ایستا، به جای ذخیره مقادیر خانه‌هایی که پاسخ پویا به آن‌ها دسترسی داشته است در جای خودشان، همه خانه‌هایی که مقدارشان تغییر کرده است را به همراه مقدار نهایی‌شان در یک دیکشنری قرار می‌دهیم، این عملیات در فضای $O(dt)$ امکان‌پذیر است. هر بار که پاسخ پویا می‌خواهد به یک خانه به‌خصوص دسترسی داشته باشد، ابتدا چک می‌کنیم که آن‌ خانه در دیکشنری وجود دارد یا خیر. اگر داشت، مقدار نهایی را برمی‌گردانیم، در غیر این صورت، $0$ برمی‌گردانیم. هر جست‌وجو $O(1)$ زمان می‌برد و نهایتا باید $O(t)$ جست‌وجو بزنیم. 
\end{proof}

در نتیجه دیدیم که کمینه برای یک پروتکل مخابره، یک کمینه برای مساله ساختمان داده ایستا است؛ همچنین، هر کمینه برای ساختمان داده ایستا، یک کمینه برای ساختمان داده پویا است. از آنجایی که در پروتکل مخابره، آلیس می‌تواند کل ورودی را برای باب بفرستد، بهترین کمینه‌ای که می‌توانیم با استفاده از این مدل از کاهش به پیچیدگی ارتباطی برای مسائل ساختمان داده به دست آوریم برابر است با $t = \Omega(\frac{\log |Q|}{\log s} )$ اگر تعداد جست‌وجوها یک چندجمله‌ای در $n$ باشد، کمینه یک مقدار ثابت خواهد بود. به همین علت، تنها مسائلی که حداقل در $n$ نمایی باشند در نظر می‌گیریم. 

\section{ کمینه}

تکنیک‌های اصلی اثبات کمینه برای مدل جست‌وجوی سلولی با استفاده از پیچیدگی ارتباطی استفاده از حذف دور\footnote{Round Elimination}، تکنیک غنا\footnote{Richness Technique}، تکنیک پیچیدگی ارتباطی حریصانه\footnote{Greedy CC Technique} و کاهش به مساله اشتراک است. در اینجا به بررسی  تکنیک حریصانه می‌پردازیم. \cite{Nisan98}

\subsection{تکنیک پیچیدگی ارتباطی حریصانه}
\begin{definition}
\textbf{شرط ردیف تک‌رنگ\footnote{Monochromatic Row}:}
یک تابع مانند $h:A\times B \rightarrow C$ را در نظر بگیرید. $A^{'} \subseteq A$ و $B^{'} \subseteq B$ شرط ردیف تک‌رنگ را ارضا می‌کنند اگر:
\begin{equation}
	\forall x \in A^{'}, \forall y,z \in B^{'}: h(x,y) = h(x,z).
\end{equation}
\end{definition}
ایده اصلی آن است که با توجه به ماتریس مخابره آلیس و باب، نشان دهیم که همه مستطیل‌هایی که شرط ردیف تک‌رنگ را ارضا می‌کنند، کوچک هستند و تا بتوانیم قضیه بعد را اعمال کنیم و یک کمینه برای تعداد دورهای مکالمه به دست آوریم. از این به بعد، به مستطیل‌هایی که شرط ردیف تک‌رنگ را ارضا کند، مستطیل خوب می‌گوییم. 

یک مساله مخابره مانند $h:A\times B \rightarrow C$ را در نظر بگیرید. آلیس ورودی $ a \in A$ و باب ورودی $b \in B$ را دارد و بنا است که مقدار $h(a,b)$ محاسبه شود. همچنین، آلیس پیام‌های خود را از مجموعه $\{1,...,s\}$ انتخاب می‌کند و باب پیام‌های خود را از مجموعه $\{1,...,k\}$ انتخاب می‌کند. 

\begin{theorem}
 اگر $h$ یک پروتکل $t$ دوری داشته باشد، یک $A^{'} \subseteq A$ و $B^{'} \subseteq B$ وجود دارد به طوری که $|A^{'}| \geq |A|/s^{t}$ و $|B^{'}| \geq |B|/k^{t}$ و  $A^{'}$ و $B^{'}$ که شرط ردیف تک‌رنگی را ارضا می‌کنند.
\end{theorem}
 
\begin{proof}
 استقرا روی $t$. برای حکم پایه، $t = 0$ در نظر بگیرید. چون هیچ مخابره‌ای انجام نمی‌شود، خروجی فقط به ورودی آلیس وابسته است، پس همه ردیف‌های باید تک‌رنگی باشند تا جوابی برای پروتکل داشته باشیم. در این‌ صورت، $A^{'} = A$ و $B^{'} = B$. 
 
 حال فرض کنید قضیه برای $t = k$ درست باشد. یک مساله ارتباطی با یک پروتکل $t+1$ دوری در نظر بگیرید. 
 برای هر  $a \in \{ 1,...,s\}$، تعریف کنید $A_{a}$ را از ورودی‌هایی مانند $x \in A$ که برای آن‌ها آلیس در ابتدا $a$ را ارسال می‌کرده است. 
 با توجه به اصل لانه کبوتری، می‌توانیم $a$ را به نحوی ثابت نگه داریم  که $|A_{a}| \geq |A|/s$. 
 همین‌طور،  تعریف کنید $B_{b}$ را از ورودی‌هایی مانند $x \in B $ که برای آن‌ها باب در ابتدا $b$ را ارسال می‌کرده است. 
 با توجه به اصل لانه کبوتری، می‌توانیم $b$ را به نحوی ثابت نگه داریم که $|B_{b}| \geq |B|/k$. 
 یک پروتکل $t$ دوری $P^{'}$ برای مساله ارتباطی محدود به $A_{a} \times B_{b}$ وجود دارد که همان پروتکل $P$ را از مرحله دوم به بعد شبیه‌سازی می‌کند. 
 با توجه به فرض استقرا، وجود دارد یک مستطیل ساخته شده از $A_{a}^{'} \times B_{b}^{'}$ به طوری که سایزشان حداقل برابر با $1/s^{t}$ و $1/k^{t}$ برابر $A_{a}$ و $B_{b}$ است. 
 در نتیجه، برابر قراردادن $A^{'} = A^{'}_{a}$ و $B^{'} = B_{b}^{'}$ جواب دلخواه را به ما می‌دهد. 
\end{proof}

\begin{example}
 کمینه  $\Omega(n)$ برای مساله پویای ارزیابی چندجمله‌ای: ساختمان داده $y \in F^{n+1}$ ضرایب یک چندجمله‌ای است و جست‌وجوی مرتبط $x \in F^{1}$ است. پاسخ آن مشخص می‌کند که آیا این نقطه عضو جندجمله‌ای است یا خیر.

 پاسخ بدیهی این  سوال $t \in O(n)$ است. برای این که این را نشان‌دهیم، لازم است یک کمینه برای مساله ارتباطی معادل آن بیابیم. 
 
 فرض کنید که باب یک چندجمله‌ای $f$ از درجه $n$ دریافت می‌کند ($y \in F^{n+1}$). 
 همچنین آلیس یک $x\in F$ دریافت می‌کند.
  هدف این مخابره، محاسبه $f(x)$ است. 
  آلیس پیام‌های خود را از مجموعه $\{1,...,s\}$ و باب پیام‌های خود را از مجموعه $\{1,...,|F|\}$ انتخاب می‌کند. 
  حالتی را فرض کنید که $F \geq 2^{n \log n}$. 
 
 حال با استفاده از ویژگی‌های چند جمله‌ای‌ها، مشخص می‌کنیم چگونه سایز مستطیل‌های خوب باید کوچک باشد. برای هر دو چندجمله‌ای متفاوت از درجه $n$، حداکثر در $n$ نقطه با هم اشتراک دارند. پس برای یک مستطیل خوب 
 $A^{'} \times B^{'}$، $|A^{'}| \leq n$ یا $|B^{'}|\leq 1$.  حال می‌خواهیم قضیه را اعمال کنیم. 
 
 اگر $|A^{'}| \leq n$، در نتیجه $n \geq |F|/s^{t}$ در نتیجه $t \geq (\log |F| - \log n) / (\log s) \in \Omega(n)$. اگر $|B^{'}| \leq 1$، پس $1 \geq (|F|^{n+1})/ |F|^{t})$. در نتیجه $t \geq n+1$ و $t \in \Omega(n)$. 
 همین پروتکل را برای حالت خاص مساله $DISJ$ در نظر بگیرید که در آن آلیس یک عضو دارد و باب یک زیرمجموعه $l$ عضوی. آلیس مجموعه پیغام‌های خود را از مجموعه $\{0,1\}^{l\log{n}}$ انتخاب می‌کند. کافی است نشان دهیم بهترین حالت این که آلیس کل ورودی را بفرستد. اگر اندازه مجموعه‌ای که آلیس از آن انتخاب می‌کند برابر با $N = 2^{n}$ باشد، ورودی آلیس $n$ بیتی و ورودی باب $N$  بیتی خواهد بود. برای یافتن مستطیل خوب لازم است توجه کنیم که دو زیر مجموعه متفاوت $l$ تایی از $N$، حداکثر در $l-1$ عضو اشتراک دارند. پس طبق حالت قبل، سایز مستطیل خوب محدود و پیچیدگی ارتباطی برابر با $n$ خواهد بود.
  \end{example}